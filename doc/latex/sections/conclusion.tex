\section{Conclusão}

Esse trabalho foi bem interessante por mostrar que da ignorância de um indivíduo sozinho pode surgir a inteligência, quando combinados de maneira a cooperarem. Apesar de ser, em teoria, mais simples do que o algoritmo de programação genética, implementado no último trabalho prático, a implementação deste trabalho foi bem trabalhosa, principalmente devido aos diversos problemas encontrados no artigo que foi usado como base para criação do enunciado.

O principal problema foi o fato de a heurística para solução do \textbf{GAP} proposta no artigo acabar, muitas vezes, gerando soluções inválidas na alocação de nós às medianas, assim gerando resultados melhores do que o possível. Fica a dúvida de se os resultados expostos na seção de avaliação de performance do artigo não são melhores apenas por conta disso, ou se esses problemas não ocorriam nos \textit{datasets} usados pelos autores. De qualquer forma, é triste ver que esse artigo tem 17 citações e o problema nunca foi notado, mesmo 12 anos depois. Por sorte, hoje em dia, principalmente na área de computação evolutiva, muitas vezes a implementação é requerida para dar mais suporte aos artigos à serem publicados.